\begin{abstract}

General Relativity is the current description of gravity. The geometry of the four-dimensional spacetime is described by Einstein’s field equations. Exact solutions to these equations describe extremely compact objects, known as black holes. The no-hair theorem states that a black hole is fully described only by its mass and spin. Black holes can be found in binaries, where two black holes orbit around each other losing energy via gravitational-wave emission. At the end of this inspiral process, the two black holes get close enough to merge into a remnant black hole. It is possible to identify three different phases of a black-hole coalescence: the inspiral, merger, and ringdown phases. 

The first direct detection of gravitational waves was made in 2015 by the ground-based interferometer LIGO and involved the coalescence of a binary black hole system. Future observations will be carried by the Laser Interferometer Space Antenna (LISA), scheduled for launch in $\sim$ 2035s. LISA will observe the coalescence of massive black hole binaries with signal-to-noise ratios (SNRs) reaching thousands. 

Gravitational waves provide a powerful tool to test General Relativity. One example is black-hole spectroscopy, which involves analysing the frequencies and damping times of gravitational waves during the ringdown phase to test General Relativity predictions, such as the no-hair theorem. A particular type of test, known as agnostic test are very useful in this context. They rely on a model that closely resembles General Relativity, with a generic deviation introduced, without requiring any assumption about the physical origin of such a deviation. 

It has been shown that the current waveform models, used to describe the gravitational wave emission from a black hole binary, are effected by systematic errors, having as result biases in the recovered parameters. In particular, Toubiana et al. \cite{toubiana2024measuringsourcepropertiesquasinormalmode} tested the pSEOBNRv5HM model by using parametrized agnostic tests. The ringdown part of the waveform is described by damped oscillations known as quasi-normal modes (QNMs). Deviations from General Relativity are introduced as fractional modifications to the QNM frequencies and damping times, which characterize the behaviour of each mode. They showed that already for sources with SNR of $\mathcal{O}(100)$, they would measure erroneous deviations due to waveform model inaccuracies.

In this thesis, we will present a modified version of the signal model with the aim to absorb biases caused by systematics. 

In the first section, we summarize some fundamental aspects of black
holes. We discuss the nature of gravitational waves, and a typical waveform produced by a black hole binary merger. After, we describe the gravitational-wave detection, focusing on the LISA interferometer. We then introduce tests of General Relativity, how this theory has been probed in the past and the current open problems we have to face, presenting also the study by Toubiana et al.

In the second section, we present our waveform modelling effort. We first introduce how the model is characterized and then how the LISA interferometer affects the signal. We also discuss how the parametrized deviations from General Relativity are introduced in the model and the modification we add to it, in the form of a variable number of sine-Gaussian wavelets. These wavelets are centred in the merger part of the signal, with the aim to compensate for the mismatch between the injected and the recovered waveform. We then present the Bayesian statical framework adopted in the analysis, how the likelihood is computed and a summary of the sampler specifications that we employ.

In the third section, we present the results of our analysis, starting from the description of the system taken into account, and the parameter estimation done on the full signal. Then, we perform the same estimation only on the inspiral portion of the signal, and use these new results as a base for the study when we allow the model to vary the fractional deviations and have a variable number of wavelets. 

Overall, we find that the inclusion of multiple wavelets in the model partially mitigates biases and improves the recovery of the General Relativity injection, but it does not fully eliminate the spurious deviations. As such, this approach alone is not sufficient for robust future tests of general relativity with LISA. Nevertheless, model selection via the Akaike Information Criterion favours models with wavelets over those without, indicating a clear, though limited, improvement. Further studies are needed to explore more realistic scenarios, such as multi-mode QNM injections, which better reflect the expected astrophysical signals, eventually taking into account new strategies to absorb systematic effects.


\end{abstract}
