\begin{abstract}

General Relativity is the current description of gravity. The geometry of the four-dimensional spacetime is described by Einstein’s field equations. Exact solutions to these equations describe extremely compact objects, known as black holes. The no-hair theorem states that a black hole is fully described only by its mass and spin. Black holes are usually found in binaries, where two black holes orbit around each other losing energy via gravitational wave emission. At the end of this inspiral process, the two black holes get close enough to merge into a remnant black hole. It is possible to identify three different phases of a black hole merger: the inspiral, merger, and ringdown phases. 

Gravitational waves are used to test General Relativity, for instance looking at their frequency during the ringdown phase. A particular type of test, known as agnostic test, allows us to test the validity of General Relativity without the necessity of making assumptions about the nature of any potential deviations from that theory. 

It has been shown, e.g by Toubiana et al. \cite{toubiana2024measuringsourcepropertiesquasinormalmode}, that the current waveform models, used to describe the gravitational wave emission from a black hole binary, are effected by systematic errors, having as result biases in the recovered parameters. In their study, they used parametrized agnostic tests, where they introduced parametrized deviation from General Relativity in the rigndown part of the waveform. 

In this thesis, we will present a modified version of the signal model with the aim to absorb biases caused by systematics. 

In the first section, we summarize some fundamental aspects of black
holes. We discuss the nature of gravitational waves, and a typical waveform produced by a black hole binary merger. After, we describe the detection of them, focusing on the LISA interferometer. We then introduce tests of General Relativity, how this theory has been probed in the past and the current open problems we have to face, presenting also the study of Toubiana et al.

In the second section, we present the waveform modelling. We first introduce how the model is characterized and then how the LISA interferometer affects the signal. We also discuss how the parametrized deviations from General Relativity are introduced in the model and the modification we add to it, in the form of a variable number of sine-Gaussian wavelets. We then present the Bayesian statical framework adopted in the analysis, how the likelihood is computed and a summary of the sampler specifications that we employ.

In the third section, we present the results of our analysis, starting from the description of the system taken into account, and the parameter estimation done on the full signal. Then, we perform the same estimation only on the inspiral portion of the signal, and use these new results as a base for the study when we allow the model to vary the fractional deviations and have a variable number of wavelets. At the end of this section, we show how this modified model fails to absorb completely the biases but, resorting to the Akaike Information Criterion, why this model should still be used to describe the waveform.

In the conclusion of the thesis, we summarize all the work done, assessing the results and suggesting way to conduct further studies on this subject.



\end{abstract}